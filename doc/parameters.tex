\section{parameters module}

In this module are defined all parameters: \textbf{L}, \textbf{N\_thermalization}, \textbf{N\_measurements}, \textbf{N\_skips}, \textbf{dt}, \textbf{epsilon}, \textbf{lambda}, \textbf{start} and \textbf{input\_file}. Contains just one subroutine called \textbf{\textit{read\_input}} that take a file named "\textit{parameters.dat}" which includes a \textit{namelist} with the parameters that the user can define for the program:

\begin{itemize}
\item \textbf{N\_thermalization}: Number of steps to the thermalization.
\item \textbf{N\_measurements}: Number of measurements.
\item \textbf{N\_skips}: Number of skips to take measurements.
\item \textbf{L}: Spaces for the lattice.
\item \textbf{dt}: Step of time.
\item \textbf{epsilon}: Range to consider new candidates at Metropolis algorithm.
\item \textbf{lambda}: Asymetry parameters.
\item \textbf{start}: How initialize the array, constant (cold) or aleatory (hot).
\end{itemize}